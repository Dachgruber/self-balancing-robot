% https://de.overleaf.com/login
% https://www.texstudio.org/

% Dokumentenklasse, Schrift- und Papiergröße
\documentclass[12pt, a4paper]{scrartcl}
% scrartcl: Kurze Aufsätze, scrreprt: Längere Arbeiten, scrbook: Buch, scrlttr2: Brief, beamer: Präsentation, ...

\usepackage{lmodern}      			% Bessere Schrift in pdf
\usepackage[utf8]{inputenc}         % Kodierung mit allen ASCII-Zeichen und den gängigsten Sonderzeichen
\usepackage[T1]{fontenc}  			% Anzeigen von Sonderzeichen und Silbentrennung
\usepackage[ngerman]{babel}			% Sprachpaket neue deutsche Rechtschreibung
\usepackage{blindtext}				% Erzeugt Blindtext zum Seitenfüllen

\usepackage{color}

% Mathe-Pakete
\usepackage{amsmath}
\usepackage{amssymb}
\usepackage{amsfonts}
\usepackage{nicefrac}				% Schräge Brüche
\usepackage{cancel}					% Zum Durchstreichen in Formeln
\usepackage{xparse}					% Wird für das Physik-Paket benötigt
\usepackage{physics}				% Physik Formeln

\usepackage{graphicx}				% Bilder
\usepackage{wrapfig}				% Textumflossene Bilder 
\usepackage{float}					% Besseres setzen von Bilden [H]
\usepackage{subfig}
\usepackage[inline]{enumitem}		% Bessere Listen
\usepackage{tabularx}				% Mehr Optionen für Tabellen
\usepackage{tabulary}				% Tabellen mit fester Gesamtbreite
\usepackage{siunitx}				% Tabellen mit Zahlen

% Verlinkungen aller Art
\usepackage[ngerman]{varioref}		% Verlinkungen mit Seite
\usepackage{hyperref}				% Verlinkungen der Kapitel
\usepackage[ngerman]{cleveref}		% Verlinkungen mit Art
% Diese Reihenfolge muss zwingend so eingehalten werden!

\usepackage{listings}				% Programmiercode setzen
% Umlaute im Quellcode
\lstset{basicstyle=\ttfamily}
\lstset{literate=%
	{Ö}{{\"O}}1
	{Ä}{{\"A}}1
	{Ü}{{\"U}}1
	{ß}{{\ss}}1
	{ü}{{\"u}}1
	{ä}{{\"a}}1
	{ö}{{\"o}}1
}

\begin{document}

\begin{titlepage}
   \begin{center}
       \vspace*{1cm}
       
       \large
		 Abschluss-Ausarbeitung \\
		 für das Modul \textbf{Elektronik für Lehrämtler} zu dem Projekt:\\
		 \vspace{0.5cm}
		\LARGE
       \textbf{ARoPra}\\
        \large
       \textbf{A}rduino \textbf{Ro}boter mit \textbf{Pr}üfungs\textbf{A}ngst\\
        \vspace{1.5cm}
		\large       
       Ein selbstbalancierender \\ Zweibein-Roboter basierend auf \\der Arduino-Platform
       
       \vspace{1.5cm}
	vorgelegt von \\
       \textbf{Cornelius Brütt}

       
       \large   
       MatNr: 1161780\\
       Mail: stu231310@mail.uni-kiel.de\\
            
       \vspace{0.8cm}
		\vfill

	\begin{figure}%
    		\centering
    		%\subfloat{{\includegraphics[width=5cm]{university_matNat} }}
    		%\qquad
    		%\subfloat{{\includegraphics[width=5cm]{img/laborino} }}
	\end{figure}
     
      \large
      
   \end{center}
   \begin{tabular}{ll}
       Seminarleitung:& Prof. Dr. Dietmar Block\\
   \end{tabular}\\
   \\
       \vspace{1.5cm}
       \noindent
       Kiel, am 05. Oktober 2023
\end{titlepage}

\end{document}